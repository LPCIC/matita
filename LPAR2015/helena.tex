\subsection{A Relevant Test Case: the ``Grundlagen'' Verified}

As a relevant test case for \elpi,
we implemented in the \frag{}
a validator for the latest version of the formal system $\lambda\delta$
\cite{lambdadeltaJ3a}, and used this validator to verify the
``Grundlagen'' \cite{Jut79} translated in a Pure Type System
\cite{Brn92}.

The formal system $\lambda\delta$ is a framework that can faithfully embed
some former typed $\lambda$-calculi including 
$Aut-QE$, the Automath dialect in which the ``Grundlagen'' was
originally written, and Pure Type Systems like $\lambda C$.

%FG: comment this paragraph in case of need
Current verification algorithms for typed systems follow
a well-established pattern prescribing a reduction
machine to compute weak head normal forms, a comparator to assert
converibility by levels, and a checker responable for type inference.
The verification algorithm for $\lambda\delta$ 
deviates slightly from this pattern in that
type checking is replaced by validation, and in that
type inference is delegated to the (extended) reduction machine. 
The performance benefits of this approach are documented in
\cite{lambdadeltaJ3a}.
We whish to recall that type checking a term means 
asserting that this term has a specified type, whereas
validating a term means asserting that this term has 
some unspecified type.

A validator for $\lambda\delta$, named Helena,
has been implemented in Caml,
and our \lp{} implementation follows it closely.
Nevertheless, the \lp{} code is much simpler that the
corresponding Caml code, and consists of just 52 clauses.

The translated ``Grundlagen'' is a theory comprising 
32 declarations and 6879 definitions, for a total of 6911 items.
Each item is a term to be verified.
Overall, the tree representation of these terms consists of
754579 nodes.
Due to this huge amount of data, the validation with Teyjus is work in
progress.
On the other hand, we can present the translated ``Grundlagen'' to Coq
\cite{lambdadeltaJ3a}.

\begin{center}
\begin{tabular}{|l|c|c|c|}
\hline
\multicolumn{4}{|c|}{User time range (s) for 31 runs}\\
\hline
Execution   & Helena              & \elpi               & Coq                 \\
\hline
compiled    & from 01.02 to 01.05 & not applicable      & from 24.26 to 24.43 \\
\hline
interpreted & \FG{TO BE MEASURED} & from 27.52 to 27.82 & from 94.18 to 95.78 \\
\hline
\end{tabular}
\end{center}
